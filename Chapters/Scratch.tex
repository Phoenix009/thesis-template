
\section{Setup}

\todo{add abbreviations for ssim, psnr, svc}

This section describes the experimental setup and methodology used to evaluate
how dropping macroblocks from layers in Scalable Video Coding
(SVC) affects video quality. The study covers both quality scalability and
spatial scalability variants. Standard quality metrics—SSIM (Structural
Similarity Index Measure) and PSNR (Peak Signal-to-Noise Ratio)—are used to
assess visual degradation.

% ------- Video Dataset --------
\subsection{Video Dataset}

\todo{cite the reference for the video dataset}

We selected four publicly available test videos from the Xiph.org video dataset
1, varying in content and motion complexity:

\begin{table}[H]
    \centering
    \begin{tabular}{l|c}
        \toprule
        Title               & \# Frames \\
        \midrule
        blue\_sky           & 217 \\
        pedestrian\_area    & 375 \\
        rush\_hour          & 500 \\
        riverbed            & 250 \\
        \bottomrule
    \end{tabular}
    \caption{Video Dataset Used for Evaluation}
    \label{tab:video_dataset}
\end{table}


\subsection{SVC Variants for Evaluation}

Each video was encoded into three SVC variants to study the effects of
enhancement layer degradation separately for each scalability type: quality,
spatial, and temporal.

\subsubsection{Only quality enhancement SVC video configuration}

For the quality scalability configuration, all layers were encoded at a fixed
resolution of 360p. The base layer used coarse quantization to provide the
minimum viable quality, while enhancement layers applied progressively finer
quantization settings to improve visual fidelity.

\begin{table}[H]
    \centering
    \begin{tabular}{l|c|c|c}
        \toprule
        Layer               & Resolution    & FrameRate     & Enhancement   \\
        \midrule
        Base Layer          & 360p          & 25fps     & Base          \\
        Enhancement Layer 1 & 360p          & 25fps     & Quality       \\
        Enhancement Layer 2 & 360p          & 25fps     & Quality       \\
        \bottomrule
    \end{tabular}
    \caption{Layer Structure for Quality Variant}
    \label{tab:layer_structure_quality}
\end{table}

\subsubsection{Only spatial enhancement SVC video configuration}

In the spatial scalability configuration, each layer increased the resolution of
the video while also improving quantization quality. The base layer was encoded
at 360p, followed by enhancement layers at 720p and 1080p.

\begin{table}[H]
    \centering
    \begin{tabular}{l|c|c|c}
        \toprule
        Layer               & Resolution    & FrameRate & Enhancement   \\
        \midrule
        Base Layer          & 360p          & 25fps     & Base         \\
        Enhancement Layer 1 & 720p          & 25fps     & Spatial      \\
        Enhancement Layer 2 & 1080p         & 25fps     & Spatial      \\
        \bottomrule
    \end{tabular}
    \caption{Layer Structure for Spatial Variant}
    \label{tab:layer_structure_spatial}
\end{table}

\subsubsection{Only temporal enhancement SVC video configuration}

In the temporal scalability configuration, all layers were encoded at a fixed
resolution of 360p, with each layer increasing the frame rate of the video. The
base layer was encoded at a low frame rate of 6fps, followed by enhancement layers with
progressively higher frame rates of 12fps and 25fps.

\begin{table}[H]
    \centering
    \begin{tabular}{l|c|c|c}
        \toprule
        Layer               & Resolution    & FrameRate & Enhancement   \\
        \midrule
        Base Layer          & 360p          & 6fps      & Base          \\
        Enhancement Layer 1 & 360p          & 12fps     & Temporal      \\
        Enhancement Layer 2 & 360p          & 25fps     & Temporal      \\
        \bottomrule
    \end{tabular}
    \caption{Layer Structure for Temporal Variant}
    \label{tab:layer_structure_temporal}
\end{table}

\section{Methodology}

Our evaluation consists of four main stages: generating layered bitstreams,
applying controlled degradation, decoding and assessing video quality, and
aggregating the results.

\todo{we need a diagram to show this methodology}

We begin by creating separate scalable video bitstreams, each representing a
different enhancement configuration. The enhancement configuration is show in Table \ref{tab:bitstreams}. 
These configurations allow us to evaluate how the presence of additional
enhancement layers affects video quality under loss.

\begin{table}[H]
    \centering
    \begin{tabular}{l|l}
        \toprule
        ID                      & Layers Included                                        \\
        \midrule
        \texttt{BL}             & BaseLayer                                              \\
        \texttt{BL\_EL1}        & Base Layer + Enhancement Layer 1                       \\
        \texttt{BL\_EL1\_EL2}   & BaseLayer + Enhancement Layer 1 + Enhancement Layer 2  \\
        \bottomrule
    \end{tabular}
    \caption{Bitstreams extracted for evaluation}
    \label{tab:bitstreams}
\end{table}


To simulate unreliable network transmission, we randomly skip macroblocks from
the top-most enhancement layer in each configuration. We vary the amount of
skipped macroblocks from 10\% to 100\%, increasing in steps of 10\%.

Each degraded version of the bitstream is decoded using the JSVM decoder. 
We then evaluate the quality of the reconstructed video using two commonly used
objective metrics: Structural Similarity Index Measure (SSIM) and Peak
Signal-to-Noise Ratio (PSNR). The original, unaltered video serves as the
reference for comparison.

To account for the randomness of macroblock drops, each experiment
is repeated multiple times per drop percentile. 
We then average the SSIM and PSNR scores across these runs to get stable quality
estimates. 
Finally, we plot the average scores against the drop percentiles for
each bitstream configuration

\section{Results}
