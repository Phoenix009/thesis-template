\section{Experimental Setup}

    \subsection{Test Videos and Encodings}
        We selected four publicly available test videos from the Xiph.org video
        dataset 1, varying in content and motion complexity. All videos are
        available in the raw YUV format (YUV420p, 1080p, 24 fps), therefore PSNR
        and SSIM values of the encoded videos can be calculated accordingly. 

        \begin{table}[H]
            \centering
            \begin{tabular}{lcr}
                \toprule
                Title               & Length        & Type      \\
                \midrule
                blue\_sky           & 217 frames    & Movie \\
                pedestrian\_area    & 375 frames    & Animation \\
                rush\_hour          & 500 frames    & Animation \\
                riverbed            & 250 frames    & Animation \\
                \bottomrule
            \end{tabular}
            \caption{Video Dataset Used for Evaluation}
            \label{tab:video_dataset}
        \end{table}

        Each video was encoded into three SVC variants, with each variant using a
        different type of scalability, to independently study the impact of
        enhancement layer loss for each type: \textbf{Quality}, \textbf{Temporal},
        and \textbf{Spatial}.

        \begin{itemize}
            \item {
                \texttt{ONLY\_QUALITY} variant uses one base layer and two quality
                enhancement layers. All layers are encoded at a fixed resolution of
                360p. The base layer employs coarse quantization to provide a
                low-quality baseline, while the enhancement layers use SNR
                scalability to progressively improve visual quality.
            } 
            \item {
                \texttt{ONLY\_SPATIAL} variant uses spatial scalability, where each
                layer increases the resolution of the video. The base layer is
                encoded at 360p, followed by enhancement layers at 720p and 1080p.
            }

            \item {
                \texttt{ONLY\_TEMPORAL} variant uses temporal scalability, where all
                layers are encoded at a fixed resolution of 360p, and each layer
                increases the frame rate of the video. The base layer is encoded at
                6\,fps, followed by enhancement layers at 12\,fps and 25\,fps.
            } 
        \end{itemize}

        \begin{table}[H]
            \centering
            \begin{tabular}{clccr}
                Configuration       & Layers            & Resolution & Frame Rate(fps) & Bitrate \\
                \midrule
                \multirow{3}{*}{\texttt{ONLY\_QUALITY}} 
                                    & BL                & 360p       & 25      & -- \\
                                    & EL1               & 360p       & 25      & Quality \\
                                    & EL2               & 360p       & 25      & Quality \\
                \midrule
                \multirow{3}{*}{\texttt{ONLY\_SPATIAL}} 
                                    & BL                & 360p       & 25      & -- \\
                                    & EL1               & 720p       & 25      & Spatial \\
                                    & EL2               & 1080p      & 25      & Spatial \\
                \midrule
                \multirow{3}{*}{\texttt{ONLY\_TEMPORAL}} 
                                    & BL                & 360p       & 6      & -- \\
                                    & EL1               & 360p       & 12      & Temporal \\
                                    & EL2               & 360p       & 25      & Temporal \\
            \end{tabular}
            \caption{Layer Structure of Test Variants}
            \label{tab:test_video_layer_structure}
        \end{table}

    \subsection{Evaluation Methodology}
        Our evaluation consists of four main stages: generating layered bitstreams,
        applying controlled degradation, decoding and assessing video quality, and
        aggregating the results.

        \begin{figure}[H]
            \centering
            \todo{we need a diagram to show this methodology}
            \caption{Evaluation pipeline overview}
            \label{fig:evaluation_pipeline}
        \end{figure}

        We begin by creating separate scalable video bitstreams, each
        representing a different enhancement configuration. The enhancement
        configuration is shown in Table~\ref{tab:bitstreams}.  These
        configurations allow us to evaluate how the presence of additional
        enhancement layers affects video quality under loss.


        \begin{table}[H]
            \centering
            \begin{tabular}{ll}
                \toprule
                Stream Id               & Stream Configuration   \\
                \midrule
                \texttt{BL}             & BL                \\
                \texttt{BL\_EL1}        & BL + EL1          \\
                \texttt{BL\_EL1\_EL2}   & BL + EL1 + EL2    \\
                \bottomrule
            \end{tabular}
            \caption{Bitstreams extracted for evaluation}
            \label{tab:bitstreams}
        \end{table}
        

        To simulate unreliable network transmission, we randomly skip
        macroblocks from the top-most enhancement layer in each configuration.
        We vary the amount of skipped macroblocks from 0-100\%, increasing
        in steps of 10\%.

        Each degraded version of the bitstream is decoded using the JSVM Decoder
        v9.19.  We then evaluate the quality of the reconstructed video using
        two commonly used objective metrics: Structural Similarity Index Measure
        (SSIM) and Peak Signal-to-Noise Ratio (PSNR). The original, unaltered
        video serves as the reference for comparison.

        To account for the randomness of macroblock drops, each experiment is
        repeated multiple times per drop percentile.  We then average the SSIM
        and PSNR scores across these runs to get stable quality estimates. 
        Finally, we plot the average scores against the drop percentiles for
        each bitstream configuration



\section{\textbf{RQ1}: Feasibility of Virtual Quality Levels in SVC}
    Discuss visual quality with dropped data

\section{\textbf{RQ2}: Frame Importance using SVC Dependencies}
    Discuss visual quality with dropped data based on frame importance

\section{Discussion of Findings}