

\section{Inter-Layer Prediction in SVC}
\label{sec:inter-layer-prediction}

    The purpose of inter-layer prediction in Scalable Video Coding (SVC) is to
    reduce the redundancy between the base layer and enhancement
    layers. This allows for higher compression efficiency without significantly
    increasing decoder complexity. These prediction tools are inspired by
    traditional single-layer prediction techniques in H.264/AVC but are extended
    to operate across layers.

    In H.264/AVC, each macroblock is coded using either intra (texture) or inter (motion)
    prediction:

    \begin{itemize}
        \item 
            \textbf{Texture prediction:} 
            Uses spatial correlation by predicting the current MB from
            neighboring MBs within the same picture.

        \item 
            \textbf{Motion prediction:} 
            Uses temporal correlation by referencing previously decoded frames.
            Motion vectors and reference indices are used to point to predictive
            blocks in reference pictures.

        \item 
            \textbf{Residual prediction:} 
            After prediction (intra or inter), the difference/residual between the
            original and predicted block is transformed,
            quantized, and coded.
    \end{itemize}

    In SVC, these same principles are extended across layers through inter-layer
    prediction, where the enhancement layer reuses information from the
    co-located MB in the base layer:

    \begin{itemize}
        \item 
            \textbf{Inter Layer Texture Prediction:} 
            Analogous to intra prediction, but instead of spatial neighbors, the
            enhancement layer MB uses the reconstructed and upsampled co-located
            MB from the base layer. This enables the MB to be coded as
            \textit{IntraBL}, saving bits by avoiding spatial prediction and
            residual transmission.

        \item 
            \textbf{Inter Layer Motion Prediction:} 
            Similar to AVC inter prediction, but the motion vectors, reference
            indices, and partitioning of the base layer MB are reused in the
            enhancement layer. For spatial scalability, motion vectors
            are upsampled.

        \item 
            \textbf{Inter Layer Residual Prediction:} 
            While AVC encodes the residual directly, SVC enhances this by
            allowing the residual of the base layer MB to be upsampled and
            subtracted from the enhancement layer residual before encoding. The
            encoder then transmits only the difference, reducing the residual
            bitrate.

    \end{itemize}

    Rate-Distortion Optimization (RDO) is used to decide whether to use these
    inter-layer modes on a per-macroblock basis, balancing bit cost against
    prediction accuracy.


\section{Macroblock Skips and Virtual Quality Levels}
\label{sec:virtual-quality-levels}
    In this section, we investigate whether intermediate quality levels beyond
    the explicitly defined scalability layers can be created by partially
    omitting enhancement-layer data.

    As introduced in Section~\ref{sec:svc-bitstream-structure}, the SVC
    bitstream is composed of access units, each containing slices corresponding
    to different enhancement layers. Each slice includes a slice header and a
    slice data section, which holds a list of macroblocks. Macroblocks are the
    smallest coding units in the video and contain motion, prediction, and
    residual data used by the decoder to reconstruct the frame.

    The slice data structure in the scalable extension is defined by the syntax
    shown in  Figure~\ref{fig:slice-data-syntax}.  A key element of this syntax
    is the \texttt{mb\_skip\_flag}, which indicates whether a macroblock is
    skipped. Macroblocks can be skipped when their content can be
    predicted from reference frames with minimal impact on visual
    quality The primary purpose of macroblock skip mode in the H.264 is to
    reduce bitrate by avoiding the transmission of redundant information.  When
    this flag is set, no motion vectors or residuals are transmitted for that
    macroblock and corresponding
    \texttt{macroblock\_layer\_in\_scalable\_extension()} syntax is omitted from
    the bitstream. Instead, the decoder reconstructs it using motion-compensated
    prediction from a reference frame. This mechanism is highly efficient, often
    requiring only a single bit to encode the skip decision, resulting in a
    reduced bitrate with minimal perceptual impact.


    This skipping mechanism applies only to macroblocks in P and B-macroblocks, which
    are inter-coded and can be predicted using motion information and residuals
    from reference frames. In contrast, intra-coded I-macroblocks rely
    entirely on spatial prediction within the same frame and do not use motion
    vectors or residuals from other frames. Since there is no external reference
    available, skipping intra-coded macroblocks is not possible, and they must
    always be fully encoded.

    \begin{figure}[H]
        \centering
        \includegraphics[width=\linewidth]{slice-data-syntax.pdf} 
        \caption{Slice data syntax in H.264/SVC}
        \label{fig:slice-data-syntax}
    \end{figure}

    In our approach, we leverage this skip mechanism to omit
    macroblocks in the enhancement layers. For each macroblock we want to skip,
    we set the \texttt{mb\_skip\_flag} and remove the associated
    \texttt{macroblock\_layer\_in\_scalable\_extension()} syntax. This allows us
    to drop enhancement layer data in a controlled way while preserving
    the decodability of the bitstream.

    By skipping macroblocks in the enhancement layers, we create
    modified versions of the original bitstream that has degraded visual quality.
    These degraded versions represent
    quality levels that are not explicitly provided by the original SVC bitstream.
    We refer to these intermediate representations as \textit{virtual quality
    levels}. Each virtual level reflects a unique quality point created by partially
    omitting enhancement data, offering finer granularity between the standard
    quality steps defined by the encoder. This allows for smoother quality
    transitions and more flexible adaptation to available network resources than
    conventional SVC layer switching.

    In Section \ref{sec:eval_virtual_quality_levels}, we evaluate these virtual quality levels using
    objective video quality metrics such as SSIM and PSNR, and assess
    their potential for improving streaming adaptability under fluctuating
    network conditions.


\section{Motion and Residual Upsampling}
\label{sec:blskip}
    Macroblock skips can produce unacceptable playback quality due to missing
    data.  In order to prevent such effects, error concealment techniques are
    highly desirable.

    Error concealment refers to the techniques used to recover missing or
    corrupted data, typically due to packet losses during transmission. SVC error concealment techniques
    rely on spatial, temporal, or inter-layer redundancy to estimate the
    lost information and maintain acceptable playback quality. While originally
    designed to address unintentional losses, such techniques are also effective in
    scenarios involving deliberate omission of data.

    In our approach, we use the \textit{Base Layer Skip (BLSkip)} error
    concealment method to reconstruct macroblocks that are skipped in
    the enhancement layer. For skipped macroblocks in the enhancement layer, BLSkip
    operates as follows:
    
    \begin{enumerate}
        \item 
        If the co-located macroblock in the base layer is
        intra-coded, inter-layer texture prediction is used to generate the enhancement
        layer content. 
        
        \item
        If the co-located macroblock in the base layer is inter-coded, both inter-layer
        motion prediction and residual prediction are applied. In this case, motion
        compensation is performed in the enhancement layer using the upsampled
        motion vectors from the base layer, and the residual signal is predicted
        accordingly. 
    \end{enumerate}
    
    This enables reconstruction of the skipped macroblock while preserving visual quality.


\section{Slice Importance}
\label{sec:slice_imp}

    \todo{add references}
    While previous sections explored macroblock-level skipping for generating
    virtual quality levels, this section introduces a complementary
    approach based on slice importance.

    The motivation for identifying slice importance stems from the observation
    that not all slices contribute equally to the perceived visual quality of a
    video.  As highlighted by Palmer et al., while some frame loss can be tolerated,
    uncontrolled (i.e. random) losses can cause severe visual artifacts and
    structural distortion in the video. Appel investigates frame-level
    importance in H.264/AVC videos by analyzing reference structures and their
    impact on visual quality under frame loss. This paper proposes several frame
    prioritization schemes that assign higher importance to frames with direct
    and transitive dependencies. The evaluation shows that dropping even a
    single referenced frame early in the decoding chain can propagate visual
    degradation across subsequent frames, reinforcing the need for a structured
    approach to data omission. In SVC, we observe a similar opportunity
    by examining the inter and intra-layer reference structures.

    In AVC, Appel identifies unreferenced B-frames as candidates for dropping,
    since their loss does not impact decoding of subsequent frames. Inspired by
    this, we analyze the direct reference count of each slice in the SVC bitstream to
    assess its importance. Specifically, we examine how often a given slice is
    directly referenced by other slices, both within the same layer (intra-layer) and
    across layers (inter-layer). A higher reference count indicates that a slice
    serves as a dependency for multiple other slices, increasing its impact on
    decoding quality. Therefore, we assign higher importance to slices with
    greater reference counts, and drop those with low or no references. This
    allows us to control quality degradation in a predictable way, ensuring that
    visually and structurally significant content is retained even under
    constrained conditions.

    By ranking slices based on this heuristic, we can establish a  retention
    priority. While macroblock skipping reduces bitrate per slice, slice
    importance controls which slices should be dropped. When degrading the
    bitstream to form virtual quality levels, low-priority slices are dropped
    first, while high-priority ones are preserved. This enables controlled
    degradation, ensuring that omitted content has minimal perceptual impact.

    To implement slice-level dropping based on the computed importance, we
    simply remove all macroblocks belonging to a selected slice from the
    bitstream. Macroblocks are dropped using the macroblock skipping approach
    described in Section \ref{sec:virtual-quality-levels}.  Importantly, we
    retain the headers and preserve I-macroblocks that are
    necessary for maintaining decodability. This ensures that the
    resulting bitstream remains decodable, even when some slices are omitted. By
    applying this method to low importance slices, we create
    virtual quality levels with controlled and targeted degradation, avoiding
    the pitfalls of random loss while offering finer granularity in quality
    control.
